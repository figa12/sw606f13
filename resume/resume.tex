\documentclass[10pt]{article}

\usepackage[utf8]{inputenc}
\usepackage[danish]{babel}
\usepackage{acronym}
\acrodef{api}[API]{Application Programming Interface}
\acrodef{cat}[CAT]{Category Adminitration Tool}
\acrodef{lamp}[LAMP]{Linux, Apache2, MySql and PHP}
\acrodef{giraf}[GIRAF]{Graphical Interface Resources for Autistic Folk}
\acrodef{oha}[OHA]{Open Handset Alliance}
\acrodef{gui}[GUI]{Graphical User Interface}
\acrodef{json}[JSON]{JavaScript Object Notation}

\acrodef{opengles}[OpenGL ES]{OpenGL for Embedded Systems}
\acrodef{pot}[POT]{power-of-two}
\acrodef{npot}[NPOT]{non-power-of-two}
\acrodef{3d}[3D]{Three-dimensional space}
\acrodef{2d}[2D]{Two-dimensional space}

\title{Interactive Learning Exercise for Children with Autism: Resumé}
\author{Jacob Karstensen Wortmann\\Jesper Riemer Andersen\\Nicklas Andersen\\Simon Reedtz Olesen}
\date{4. juni 2013}

\begin{document}
\maketitle

Vores applikation, Train, blev udviklet som en del af \ac{giraf} systemet, som blev startet i 2011. \ac{giraf} systemet er et multiprojekt som i 2013 bestod af otte grupper. Visionen for \ac{giraf} er at lave en multifunktionel applikation til Android platformen som kan gøre livet nemmere for børn med autisme, deres pædagoger og forældre. Formålet er at erstatte fysiske genstande, som hjælper dem i dagligdagen, med digitaliseret udgaver. Ideen er at samle flere funktionaliteter et sted og tillade rig mulighed for konfiguration.

Vores gruppes fokusområde var at udvikle et spil til \ac{giraf} systemet. Vi valgte at tage udgangspunkt i en øvelse som vores kontaktperson, Tove Søby, allerede udøver med børnene. Øvelsen går ud på at associere pictogrammer og/eller objekter med hinanden. For at gøre det til en leg havde hun et papir-tog til at køre rundt med objekterne.

Train applikationen består af to dele, en menu til at oprette spil-konfigurationer, og selve spillet. 

Når applikationen starter, ser man menuen, her har man mulighed for at vælge det barn man ønsker at lave en spil-konfiguration for. En spil-konfiguration består af et antal stationer, som hver har en kategori og en mængde associerede pictogrammer. Når man har lavet en spil-konfiguration, kan man starte spillet.

I spillet følger man toget fra station til station og sætter de associerede pictogrammer af på hver station. Når der ikke er flere pictogrammer tilbage i toget så er spillet gennemført.

Den grafiske del af spillet er implementeret ved hjælp af \acl{opengles}. Vi gør brug af drag-and-drop for at flytte pictogrammer til og fra toget.

\end{document}