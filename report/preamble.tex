% Define document class. Important.
\documentclass[a4paper,twoside,openright]{report}% MANGLER 'openright' HER

% Margener
%\setlength{\evensidemargin}{1cm}
%\setlength{\oddsidemargin}{1cm}

% Set up encoding
\usepackage[utf8]{inputenc}


% Load up bibliography.
\usepackage[authoryear]{natbib}
\setcitestyle{numbers,square}
% Bibliography style.
\bibliographystyle{plainnat}

% Algorithm support.
\usepackage{algorithmic}
\usepackage{algorithm}
\usepackage{subfig}
\usepackage{amsmath}
\usepackage{amsfonts}
% Make algorithms appear as procedures instead.
\floatname{algorithm}{Procedure}
\renewcommand{\algorithmicrequire}{\textbf{Input:}}
\renewcommand{\algorithmicensure}{\textbf{Output:}}

% Image frames.
\setlength{\fboxsep}{0pt}
\setlength{\fboxrule}{0.5pt}

% Also, images.
\usepackage{graphicx}

% tabeller der strækker sig over flere sider
\usepackage{longtable}

% flere tabel-muligheder
\usepackage{multirow}

% bedre enumerate
\usepackage{enumitem}

% Todo notes here and there.
% write instead for disable: \usepackage[disable]{todonotes}
\usepackage{todonotes}

% Forbedrede floats.
\usepackage{float}
\usepackage{rotating}

% Special symbols availability.
\usepackage{amssymb}

%Degree symbol
\usepackage{gensymb}

% Wrap figure
\usepackage{wrapfig}

% Operationel semantik
\newcommand{\lag}{\langle}
\newcommand{\rag}{\rangle}
\newcommand{\setof}[2]{\ensuremath{\{ #1 \mid #2 \}}}
\newcommand{\set}[1]{\ensuremath{\{ #1 \}}}
\newcommand{\besk}[1]{\ensuremath{\lag #1 \rag}}
\newcommand{\ra}{\rightarrow}
\newcommand{\lra}{\longrightarrow}
\newcommand{\Ra}{\Rightarrow}

% CODE %
\usepackage{listings}
\usepackage{color}
%\usepackage{bera}
\definecolor{gray}{rgb}{0.4,0.4,0.4}
\definecolor{darkblue}{rgb}{0.0,0.0,0.6}
\definecolor{cyan}{rgb}{0.0,0.6,0.6}
\lstset{
  basicstyle=\ttfamily,
  columns=fullflexible,
  showstringspaces=false,
  commentstyle=\color{gray}\upshape,
  basicstyle=\small,
  numberstyle=\footnotesize,
  numbers=left,
  captionpos=b,
  stepnumber=1,
  numbersep=10pt,
  tabsize=2,
  breaklines=true,
}
% Define markup of XML
\lstdefinelanguage{XML}
{
  morestring=[b]",
  morestring=[s]{>}{<},
  morecomment=[s]{<?}{?>},
  identifierstyle=\color{darkblue},
  keywordstyle=\color{cyan},
  morekeywords={id, target, type, category, value, point, correct, rows, width, time}% list your attributes here
}
% Define markup of C#
\lstdefinelanguage{CSharp}[Visual]{C++}
{
	identifierstyle=\color{darkblue},
	commentstyle=\color{green!70!black}\itshape ,
	stringstyle=\color{gray},
	sensitive=true,
	morestring=[b]",
	morestring=[b]',
	morecomment=[l]//,
	morecomment=[n]{/*}{*/}
}

% Define markup of Javascript
\lstdefinelanguage{JavaScript}{
  keywords={typeof, new, true, false, catch, function, return, null, catch, switch, var, if, in, while, do, else, case, break},
  keywordstyle=\color{blue}\bfseries,
  ndkeywords={class, export, boolean, throw, implements, import, this},
  ndkeywordstyle=\color{darkgray}\bfseries,
  identifierstyle=\color{black},
  sensitive=false,
  comment=[l]{//},
  morecomment=[s]{/*}{*/},
  commentstyle=\color{purple}\ttfamily,
  stringstyle=\color{red}\ttfamily,
  morestring=[b]',
  morestring=[b]"
}

% Define markup of JSON
\colorlet{punct}{red!60!black}
\definecolor{background}{HTML}{EEEEEE}
\definecolor{delim}{RGB}{20,105,176}
\colorlet{numb}{magenta!60!black}
\lstdefinelanguage{json}{
    basicstyle=\normalfont\ttfamily,
    numbers=left,
    numberstyle=\scriptsize,
    stepnumber=1,
    numbersep=8pt,
    showstringspaces=false,
    breaklines=true,
    frame=lines,
    backgroundcolor=\color{background},
    literate=
     *{0}{{{\color{numb}0}}}{1}
      {1}{{{\color{numb}1}}}{1}
      {2}{{{\color{numb}2}}}{1}
      {3}{{{\color{numb}3}}}{1}
      {4}{{{\color{numb}4}}}{1}
      {5}{{{\color{numb}5}}}{1}
      {6}{{{\color{numb}6}}}{1}
      {7}{{{\color{numb}7}}}{1}
      {8}{{{\color{numb}8}}}{1}
      {9}{{{\color{numb}9}}}{1}
      {:}{{{\color{punct}{:}}}}{1}
      {,}{{{\color{punct}{,}}}}{1}
      {\{}{{{\color{delim}{\{}}}}{1}
      {\}}{{{\color{delim}{\}}}}}{1}
      {[}{{{\color{delim}{[}}}}{1}
      {]}{{{\color{delim}{]}}}}{1},
}

\lstdefinelanguage{KAPAOOW}{
 sensitive=false,
 keywords={character, characters, action, end, if, then, else, from, to, downto, next, while, loop, use, turn, begins, ends, select, wins, draw, random, of, case, cases, enemy, player, start, skip, attack, types, damage, defend, by, using, message, and, or, is, value, mod},
 identifierstyle=\itshape,
 keywordstyle=\bfseries,
 stringstyle=\normalfont,
 morestring=[b]",
 comment=[l]{//},
 commentstyle=\color{gray}
}

% Neat-o referencer...o.
\usepackage{bookmark,hyperref}
\usepackage{nameref}

% hack fra nettet.
% http://tex.stackexchange.com/questions/1230/reference-name-of-description-list-item-in-latex
\makeatletter
\let\orgdescriptionlabel\descriptionlabel
\renewcommand*{\descriptionlabel}[1]{
  \let\orglabel\label
  \let\label\@gobble
  \phantomsection
  \edef\@currentlabel{#1}
  %\edef\@currentlabelname{#1}
%  \let\label\orglabel
  \orgdescriptionlabel{#1}
}
\makeatother
% Rettehak. Meget lettere end \checkmark
\newcommand{\yes}{\checkmark}

% Let's put in a lot of niceness in the display, yeh?
\usepackage{fancyhdr} % Get some niceness into our headers.
\pagenumbering{arabic} % Ensure page numbering in our desired form.
\pagestyle{fancy}
% Page design from fancyhdr.
\fancyhead{}
\fancyfoot{}
\fancyhead[RO,LE]{\leftmark\\\rightmark}% afsnit information i toppen af sider, over stregen. 
\fancyfoot[C]{\thepage}
\setlength{\headheight}{23pt}

% Rewrite header and footer commands.
\renewcommand{\headrulewidth}{1.0pt}
\renewcommand{\footrulewidth}{1.0pt}

% Create a new command, HRule, to insert some nice horisontal rules on the title page.
\newcommand{\HRule}{\rule{\linewidth}{0.3mm}}

% New command for two figures, side by side.
\newcommand{\twofigs}[6]
{
	\begin{figure}[H]
		\begin{minipage}[b]{0.5\columnwidth}
		\centering
		\includegraphics[width=0.8\columnwidth]{img/#1}
		\caption{#2\label{#3}}
		\end{minipage}
		\hspace{0.5cm}
		\begin{minipage}[b]{0.5\columnwidth}
		\centering
		\includegraphics[width=0.8\columnwidth]{img/#4}
		\caption{#5\label{#6}}
		\end{minipage}
	\end{figure}
}

% Sørg for at paragrafplads ikke spildes.
\raggedbottom
\usepackage[titletoc]{appendix}

% Package til at regne forskellen ud mellem 2 labels
\usepackage{refcount}
\newcommand{\pagedifference}[2]{\number\numexpr\getpagerefnumber{#2}+1-\getpagerefnumber{#1}\relax}

