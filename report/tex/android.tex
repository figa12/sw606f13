\todo{This Chapter describes the Android platform, and which key functionality of the Android platform that we have used.}
\section{Drag and Drop}
\label{sec:androiddraganddrop}
This section will explain how drag and drop works in version Android SDK 11 or greater\citep{androiddraganddropguide}. It will explain how to start a drag, what happens when view is dragged on the screen, and finally what happens when the view is dropped.
\subsection*{Overview}
Before we begin explaining drag and drop we first need to introduce the terms Listener, View and ViewGroup.
\begin{description}
\item[View] is a base class for user interface components. A View represents a square on the device's screen, and is responsible for drawing and event handling. A example of the View class could be a ImageView, which is used to display a image.

\item[ViewGroup] is a subclass of the View class, and is a base class for all layouts, which are invisible containers for Views or other ViewGroups.

\item[Listeners] are attached to a View and is mostly used to receive and handle user input. Every Listener has a\todo{Hvad kan den kaldes? OnX() method} \lstinline|OnSomething()| method, this method is called when the system receive a gesture on the View. A simple example of a Listener is the OnTouchListener, The \lstinline|OnTouch()| method is called every time the OnTouchListener's View receives a touch gesture.
\end{description}

\subsection*{Drag Shadow}
Before we can make a drag we must first create something called a drag shadow. A drag shadow is basically a copy of the View you are trying to drag which follows the position of the finger.\autoref{lst:makedragshadow} shows how to create drag shadow and start dragging a View.
\begin{lstlisting}[language=java,firstnumber=1,caption={How to create a drag shadow, and start drag},label=lst:makedragshadow]
imageView.setOnTouchListener(new OnTouchListener(){
	public boolean onTouch(View view, MotionEvent motionEvent){
		if (motionEvent.getAction() == MotionEvent.ACTION_DOWN) {
			ClipData data = ClipData.newPlainText("", "");
			DragShadowBuilder shadow = new View.DragShadowBuilder(view);
			view.startDrag(data, shadow, null, 0);
			return true;
		}
		else {
			return false;
		}	
	}
});
\end{lstlisting}
\begin{description}
\item[Line 1] set a \lstinline|OnTouchListener| on the View \lstinline|imageView|
\item[Line 2 \& 3] The \lstinline|OnTouch()| method is called by the system when the \lstinline|imageView| receives a input gesture, and if the gesture is a press motion it will continue.
\item[Line 4 \& 5] \lstinline|ClipData| is used to store text data about the view being dragged. \lstinline|shadow| is initialized with the touched \lstinline|imageView|.
\item[Line 6] The \lstinline|view.startDrag()| starts the drag of the \lstinline|imageView|. This will cause OnDragListeners of every View to be called. 
\item[Line 7 \& 10] The \lstinline|OnTouch()| method returns either true or false based on if the motion event was handled or not.
\end{description}
\subsection*{Drag Events}
A OnDragListener are like the OnTouchListener also attached to a View class. The drag of a View begins when the method \lstinline|view.startDrag()| is called. This will cause the system to draw the drag shadow, and call every OnDragListener with an action. During a drag operation it is possible to enter these four states.
\begin{description}
\item[Started] The OnDragListerner receives a drag event with the action \lstinline|ACTION_DRAG_STARTED|. If the OnDragListener returns true it will continue to receive drag events, if it return false, it will stop receiving drag events, and the view will not be able to accept the dragged data.

\item[Dragging] If the user continues to drag the shadow and enters a View's boundaries, then the system will send a drag event to the View's OnDragListener with the action \lstinline|ACTION_DRAG_ENTERED|. And when the drag shadow exits a View's boundaries the system will send a drag event with the action \lstinline|ACTION_DRAG_EXITED|.

\item[Dropped] If the user releases the drag shadow over a View, the View's OnDragListener receives a drag event with the action \lstinline|ACTION_DROP|. However the drag event is not sent if the OnDragListener previously returned false in the start state. The OnDragListener is expected to return true if the drop was successfully processed, and otherwise false.

\item[Ended] After the user has released the drag shadow, all the OnDragListeners which still are registered to receive drag events,  receives a drag event with the action \lstinline|ACTION_DRAG_ENDED|. This event is done regardless of where the drag shadow has been released.
\end{description}
An implementation and use of drag and drop is shown later in \secref{sec:implementationdraganddrop}.

\section{Listview}
The menu for customisation in this project consists of three lists and it would thus be obvious to use the ListView class from the android library\citep{androidlistview}. This proved to be quite the challenge, since the ListView is not optimised for being manipulating directly. This means that it is build for showing a data set in a list form, but if you want to change this data set on the fly\todo{runtime? or at any given time}, you run into a lot of problems. There are plenty of examples of people who struggle to use the ListView class\citep{listviewfail} and what seemed to cause a lot of our problems is the NotifyDataSetChanged method from the underlaying adapter\citep{notifydatasetchanged}. The documentation of this method is very vague and we could not get our list to update with new data.\\
We chose to implement our own classes for populating the lists containing saved game configurations and stations. The first list in the menu, the one containing children, is made with the ListView class, this could be done since it does not have to be manipulated after the initial load of children.
\subsection*{GameLinearLayout}
The class we made to show a list of saved games is called GameLinearLayout.\\

\begin{lstlisting}[language=java,firstnumber=1,caption={The method to create a list item},label=lst:makeview]
private void makeView(GameConfiguration gameConfiguration) {
        LayoutInflater layoutInflater = (LayoutInflater) getContext().getSystemService(Context.LAYOUT_INFLATER_SERVICE);
        /* Use same layout style as the profile list */
        View gameListItem = layoutInflater.inflate(R.layout.game_list_item, null);
        
        TextView gameNameTextView = (TextView) gameListItem.findViewById(R.id.gameName);
        gameNameTextView.setText(gameConfiguration.getGameName());

        ImageView gameIconImageView = (ImageView) gameListItem.findViewById(R.id.gameIcon);
        
        Bitmap bitmap = BitmapFactory.decodeFile(PictoFactory.INSTANCE.getPictogram(super.getContext(),gameConfiguration.getStation(0).getCategory()).getImagePath());
        gameIconImageView.setImageBitmap(bitmap);
        
        gameListItem.setOnClickListener(new OnItemClickListener(gameConfiguration));
        gameListItem.setOnLongClickListener(new OnItemLongClickListener(gameConfiguration));

        /* Add to the list of visible configurations */
        this.visibleGameConfigurations.add(gameConfiguration);
        super.addView(gameListItem);
    }
\end{lstlisting}
\begin{description}
\item[Line 1] \lstinline|makeView| takes a \lstinline|GameConfiguration| which contains all relevant information about the saved game.
\item[Line 2-4] Sets the style of the items. Same as the profile list containing children.
\item[Line 6-7] Gets the name of the game and sets it in the list item.
\item[Line 9-13] Gets the category pictogram of the first station in the game configuration and sets this as a game icon for the list item.
\item[Line 14-15] Set a LongClickListener and a ClickListener on the item. These are both ClickListeners that we have made, that takes a GameConfiguration. The normal click will select the item and show its details in the customisation list. The long click will show a dialog that asks you whether you want to delete it or not. 
\item[Line 19-20] Add the item to the list of saved games.
\end{description}

\subsection*{CustomiseLinearLayout}
The class we made to show a list of stations. The customisation list.

\begin{lstlisting}[language=java,firstnumber=1,caption={The method to add a station to the list},label=lst:addstation]
public void addStation(StationConfiguration station) {
        this.stations.add(station);
        this.preventStationOverflow();
        
        LayoutInflater layoutInflater = (LayoutInflater) getContext().getSystemService(Context.LAYOUT_INFLATER_SERVICE);
        View stationListItem = layoutInflater.inflate(R.layout.station_list_item, null);
        
        PictogramButton categoryPictogramButton = (PictogramButton) stationListItem.findViewById(R.id.list_category);
        categoryPictogramButton.bindStationAsCategory(station);
        
        /* The order og image button and associated pictograms layout statements, are very important here */
        ImageButton addPictogramsButton = (ImageButton) stationListItem.findViewById(R.id.addPictogramButton);
        this.addPictogramButtons.add(addPictogramsButton);
        
        AssociatedPictogramsLayout associatedPictogramsLayout = (AssociatedPictogramsLayout) stationListItem.findViewById(R.id.associatedPictograms);
        this.associatedPictogramsLayouts.add(associatedPictogramsLayout);
        associatedPictogramsLayout.bindStation(station);
        
        addPictogramsButton.setOnClickListener(new AddPictogramsClickListener(associatedPictogramsLayout));
        
        ImageView deleteButton = (ImageView) stationListItem.findViewById(R.id.deleteRowButton);
        deleteButton.setOnClickListener(new RemoveClickListener(station));
        
        super.addView(stationListItem);
    }
\end{lstlisting}
\begin{description}
\item[Line 3] The \lstinline|preventStationOverflow| method will disable the "add station" button if the max allowed amount of stations has been added.
\item[Line 5-6] Sets the style of the list items.
\item[Line 8-9] Sets the button that holds the category. This is bound to the station, which means that whenever this button changes pictogram, the category of the station changes category accordingly.
\item[Line 12-13] Sets the button to add pictograms.
\item[Line 15-17] Sets the associated pictograms layout in the list item. This is also bound to the station.
\item[Line 19] Sets an \lstinline|OnClickListener| for the \lstinline|addPictogramsButton| that will launch the \ac{cat} application when clicked. This will allow you to select the pictograms you want to add to the station.
\item[Line 21-22] Sets a delete button on the list item, that will delete the station and list item if clicked.
\item[Line 24] Adds the list item to the list.
\end{description}

