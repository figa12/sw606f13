The GIRAF project is a multi-project consisting of eight groups each with their own sub-project. The GIRAF system is developed for the Android platform, and our focus was to develop a game. 

The application is inspired by an exercise that our contact person, Tove Søby, uses with the children with autism that she works with. The application makes use of drag and drop to drag pictograms to fulfil the games' criteria. 

The application is split in two parts, the graphical part and a game customisation menu. The graphical part is developed using the \ac{opengles} graphics API. 

The customisation menu allows for easy creation, saving, and deleting of customised games, each individually child can have unique customised games saved to their profile. 