\chapter{Design}

\todo{This Chapter describes the design of our application. It is split into two sections, one describing the design of our game and the other describing the design of menu.}

\section{Game}
\todo{This section describes the design choices made in our game.}


\section{Menu}
\todo{This section describes the design choices made when creating our menu.}

\subsection{Flow}
To show the functionalities of the menu, the natural flow will be shown in this section.
\begin{figure}[H]
\centering
\includegraphics[width=0.9\linewidth]{img/screenshots/profile_flow_1.jpg}%0.1 margin
\caption{Profile flow 1}
\label{fig:profile_flow_1}
\end{figure}
\ref{fig:profile_flow_1} shows the general design of the menu. The overall color theme (Orange in this example) can be changed from the Launcher. The menu is divided into three parts; profiles, saved games, and game customisation. When the application is launched the menu will appear and the profile list will be populated with the children associated to the current guardian. In this example "Emil" has been chosen, but there are no saved games related to that child, yet.
We will now continue to push the "Tilføj station" (Add station) button three times.

\begin{figure}[H]
\centering
\includegraphics[width=0.9\linewidth]{img/screenshots/profile_flow_2.jpg}%0.1 margin
\caption{Profile flow 2}
\label{fig:profile_flow_2}
\end{figure}

The game customisation part of the menu has now been populated with three elements, each representing a station. An element/station consists of three buttons:
\begin{itemize}
\item \textbf{Add category:} Pressing the category pictogram will allow you to choose a different one. It is blank on default to show that no category has been selected.
\item \textbf{Add pictograms:} Pressing the green "+" will allow you to select the pictograms you want to be associated to the category.
\item \textbf{Delete station:} Pressing the red "x" will delete the station.
\end{itemize}
We will now continue to select a category for the first station by pressing the add category button.

\begin{figure}[H]
\centering
\includegraphics[width=0.9\linewidth]{img/screenshots/profile_flow_3.jpg}%0.1 margin
\caption{Profile flow 3}
\label{fig:profile_flow_3}
\end{figure}

\ref{fig:profile_flow_3} shows that CAT \todo{Acronym for CAT, go!} opens to allow you to select a pictogram to use for the category.
The pictogram "House" has been selected in this example, and we will now continue to press the "Send" button. This will send the pictogram back to the menu.

\begin{figure}[H]
\centering
\includegraphics[width=0.9\linewidth]{img/screenshots/profile_flow_4.jpg}%0.1 margin
\caption{Profile flow 4}
\label{fig:profile_flow_4}
\end{figure}

It can now be seen in \ref{fig:profile_flow_4} that the pictogram we selected before has been set as the category for the first station.

\begin{figure}[H]
\centering
\includegraphics[width=0.9\linewidth]{img/screenshots/profile_flow_5.jpg}%0.1 margin
\caption{Profile flow 5}
\label{fig:profile_flow_5}
\end{figure}

\begin{figure}[H]
\centering
\includegraphics[width=0.9\linewidth]{img/screenshots/profile_flow_6.jpg}%0.1 margin
\caption{Profile flow 6}
\label{fig:profile_flow_6}
\end{figure}

\begin{figure}[H]
\centering
\includegraphics[width=0.9\linewidth]{img/screenshots/profile_flow_7.jpg}%0.1 margin
\caption{Profile flow 7}
\label{fig:profile_flow_7}
\end{figure}

\begin{figure}[H]
\centering
\includegraphics[width=0.9\linewidth]{img/screenshots/profile_flow_8.jpg}%0.1 margin
\caption{Profile flow 8}
\label{fig:profile_flow_8}
\end{figure}