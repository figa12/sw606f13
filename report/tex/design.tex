\section{Menu}
\todo{This section describes the design choices made when creating our menu.}

\subsection{Flow}
To show the functionalities of the menu, the natural flow will be shown in this section.
\begin{figure}[H]
\centering
\includegraphics[width=0.9\linewidth]{img/screenshots/profile_flow_1.jpg}%0.1 margin
\caption{The main activity}
\label{fig:profile_flow_1}
\end{figure}
\ref{fig:profile_flow_1} shows the general design of the menu. The overall color theme (Orange in this example) can be changed from the Launcher. The menu is divided into three parts; profiles, saved games, and game customisation. When the application is launched the menu will appear and the profile list will be populated with the children associated to the current guardian. In this example "Emil" has been chosen, but there are no saved games related to that child, yet.
We will now continue to push the "Tilføj station" (Add station) button three times.
\todo{TESTETSTESTESTESTESTESTETSTESTETSTESTETSTESTETSTETSTETSTETSTETSTETSTETSTETSTETSTETSTETSTET}

\begin{figure}[H]
\centering
\includegraphics[width=0.9\linewidth]{img/screenshots/profile_flow_2.jpg}%0.1 margin
\caption{The main activity with three stations}
\label{fig:profile_flow_2}
\end{figure}

The game customisation part of the menu has now been populated with three elements, each representing a station. An element/station consists of three buttons:
\begin{itemize}
\item \textbf{Add category:} Pressing the category pictogram will allow you to choose a different one. It is blank on default to show that no category has been selected.
\item \textbf{Add pictograms:} Pressing the green "+" will allow you to select the pictograms you want to be associated to the category.
\item \textbf{Delete station:} Pressing the red "x" will delete the station.
\end{itemize}
We will now continue to select a category for the first station by pressing the add category button.

\begin{figure}[H]
\centering
\includegraphics[width=0.9\linewidth]{img/screenshots/profile_flow_3.jpg}%0.1 margin
\caption{PictoAdmin with one pictogram selected}
\label{fig:profile_flow_3}
\end{figure}

\ref{fig:profile_flow_3} shows that CAT \todo{Acronym for CAT, go!} opens to allow you to select a pictogram to use for the category.
The pictogram "House" has been selected in this example, and we will now continue to press the "Send" button. This will send the pictogram back to the menu.

\begin{figure}[H]
\centering
\includegraphics[width=0.9\linewidth]{img/screenshots/profile_flow_4.jpg}%0.1 margin
\caption{The main activity with one category chosen}
\label{fig:profile_flow_4}
\end{figure}

It can now be seen in \ref{fig:profile_flow_4} that the pictogram we selected before has been set as the category for the first station. We will now press the green "+" to add the pictograms we want associated with this category.

\begin{figure}[H]
\centering
\includegraphics[width=0.9\linewidth]{img/screenshots/profile_flow_5.jpg}%0.1 margin
\caption{The main activity with one complete station}
\label{fig:profile_flow_5}
\end{figure}

It can be seen in \ref{fig:profile_flow_5} that two pictograms was selected, again via CAT. \todo{Acronym for CAT, go!} We will proceed to add categories and pictograms to the remaining stations, using the same method.

\begin{figure}[H]
\centering
\includegraphics[width=0.9\linewidth]{img/screenshots/profile_flow_6.jpg}%0.1 margin
\caption{The main activity with three complete station}
\label{fig:profile_flow_6}
\end{figure}

\ref{fig:profile_flow_6} shows that the three stations have now been filled with pictograms. Note that the green "+" is gone, this is because we only allow a total of six pictograms. We will now press the "Gem nyt spil" button to save the game.

\begin{figure}[H]
\centering
\includegraphics[width=0.9\linewidth]{img/screenshots/profile_flow_7.jpg}%0.1 margin
\caption{Save dialog}
\label{fig:profile_flow_7}
\end{figure}

\ref{fig:profile_flow_7} shows that a dialog appears when the save game button has been pressed. This dialog allows you to enter a name for your game and save it. We will call our game "Our game" and press "Gem nyt spil" which will save the game.

\begin{figure}[H]
\centering
\includegraphics[width=0.9\linewidth]{img/screenshots/profile_flow_8.jpg}%0.1 margin
\caption{Complete example of the main activity}
\label{fig:profile_flow_8}
\end{figure}

The game configuration has now been saved, and is shown in the list of saved games. See \ref{fig:profile_flow_8}.\\
The newly saved game is linked to "Emil", and will show whenever he is selected.
