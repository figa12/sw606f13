Our application was developed as a part of the GIRAF project 2013, a multi-project consisting of eight project groups, each with their own sub-project. Our focus in this project was to develop a game for GIRAF. 

The problem definition in \chapref{chap:intro} states:

\begin{quote}
\textit{"In what ways can we aid the pedagogues in their work with children with autism, by digitalizing a physical exercise onto an Android tablet?"} 
\end{quote}
The exercise we chose was an exercise that our contact person, Tove, had presented during the first week of the project. The child had to place pictograms onto a train and unload them at the correct stations, this already allowed for great customization but required a lot of work from Tove, so it was an ideal exercise for us to try and implement this on the Android platform. 

The main focus with our application was to make it customisable for the guardian, so that they easily can create new and different games for each child. 

We achieved this by creating a menu using linear layouts so they are possible to edit on runtime. The guardian is able to choose a specific child from the list and start creating a new game for that child. Using \ac{cat} they are able to choose the pictogram they want as category for each station and what pictograms they want associated with each station. This allows for customisation with very limited effort from their side, which is what we wanted to achieve. 

The different graphical elements in the graphical side of the application is drawn in vector graphics, this makes it possible to resize the graphics in the future without a loss of quality. 

We have also implemented a whistle sound when the train starts driving and different random elements that can occur throughout the game, such as cows and trees on the hills in the background. This was done to create an element of surprise, which is something Tove mentioned the children really liked. 

Throughout the project we have communicated with Tove for feedback and ideas and we have shown her and had her try out the application, however we do not have any official usability tests to document that our application is satisfactory, but we have managed to develop an application to aid the pedagogues in their work with autistic children, which was our goal for this project. 