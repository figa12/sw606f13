\section{Train \& Wagons}
The train and wagons are stationary textures on the tablet screen, it is actually the background that moves in order to create the illusion that the train is driving. \autoref{lst:loadtexture} shows how we load the texture for our train and wagons.

\begin{lstlisting}[language=java,firstnumber=1,caption={Loading the texture for our train and wagons.},label=lst:loadtexture] 
	// Initialize a train object and a wagon object
	private final Texture train = new Texture(1.0f, 1.0f);
	private final Texture wagon = new Texture(1.0f, 1.0f);

	//Load the textures
	this.wagon.loadTexture(R.drawable.texture_wagon, AspectRatio.BitmapOneToOne);
	this.train.loadTexture(R.drawable.texture_train, AspectRatio.BitmapOneToOne); 
\end{lstlisting}

\begin{description}
\item[Lines 2-3] Initialising the \lstinline|train| and \lstinline|wagon| objects, with the size $1 \times 1$.
\item[Line 6] This loads the \lstinline|wagon| texture, \lstinline|R.drawable.texture_wagon| defines which texture gets loaded, which in this case is the wagon and \lstinline|AspectRatio.BitmapOneToOne| makes sure the texture keeps the original aspect ratio by resize the texture, as described in \secref{sec:openglimp}.
\item[Line 7] This loads the \lstinline|train| texture, and the original aspect ratio is kept again.
\end{description}

Now the texture is successfully loaded we have to place it on right location on the tablet screen. This is done using the coordinate system mentioned in \secref{sec:frustum}. This is done as shown in \autoref{lst:addcoordinate}.

\begin{lstlisting}[language=java,firstnumber=1,caption={Placing the texture on the screen.},label=lst:addcoordinate] 
	//Add coordnates to the renderables
	this.wagon.addCoordinate(-542.32f, -142.72f, GameData.FOREGROUND);
	this.wagon.addCoordinate(-187.45f, -142.72f, GameData.FOREGROUND);
	this.train.addCoordinate(160.42f, -52.37f, GameData.FOREGROUND);
\end{lstlisting}

\begin{description}
\item[Lines 2-3] We have chosen to have two wagons on our train, but we only have one wagon object with two coordinates. \lstinline|GameData.Foreground| determines the depth in the frustum where the objects are placed.
\item[Line 4] Adds a coordinate for the placement of the \lstinline|train|.
\end{description}

Now the textures have been loaded and given coordinates, they are ready to be drawn. This is done in \autoref{lst:drawtexture}

\begin{lstlisting}[language=java,firstnumber=1,caption={Drawing the texture on the screen.},label=lst:drawtexture] 
	//Drawing the textures
	super.translateAndDraw(this.wagon);
	super.translateAndDraw(this.train);
\end{lstlisting}

\begin{description}
\item[Lines 2-3] This draws the \lstinline|wagon| and \textit{train} objects. The \lstinline|translateAndDraw()| method is explained in \secref{sec:gamerendering}
\end{description}

\subsection{Wheels}

To create the illusion that the train is moving, we had to make wheels rotate in order to make it look like it was actually driving. 

The wheels are loaded and given a coordinate in the same way as the train that was just explained.

The difference comes when the wheels are drawn, we have to rotate the wheels so that it looks like the train moves, this is done by calculating the rotation using the function shown in \autoref{lst:calcrotate}.

\begin{lstlisting}[language=java,firstnumber=1,caption={Calculating the wheel rotation.},label=lst:calcrotate]
    private float[] rotation = { 0f, 0f, 0f }; // rotation number for each wheel size
    private final double[] wheelDiameter = {
            106.39f, // large wheel
            78.71f,  // medium wheel
            60.8f    // small wheel
    };

    private final float calculateRotation(int wheelIndex) {    
        float circumference = this.wheelDiameter[wheelIndex] * Math.PI;
        float degreePerPixel = 360.0 / circumference;
        this.rotation[wheelIndex]+= degreePerPixel * super.gameData.getPixelMovement();
        return this.rotation[wheelIndex];
    }
\end{lstlisting}

\begin{description}
\item[Line 1] This array has the rotation number for each wheel size.
\item[Lines 2-6] This array has the diameter in pixels for each wheel size. 
\item[Line 8] The function takes a wheelIndex as parameter, this is to determine which the wheel size that is being used for calculations. 
\item[Line 9] The wheel's circumference is being calculated.
\item[Line 10] Calculating the degreePerPixel by dividing 360 with the circumference
\item[Line 11] Here the rotation for the specific wheel is calculated by taking the degreePerPixel we just found and multiplying it by the pixel movement. \lstinline|GameData.getPixelMovement| is explained in \secref{sec:gamerendering}. Please note that the rotation is not reset, it keeps getting added to, however is it is not necessary to reset it as the rotation will not reach an overflow.
\item[Line 12] The specific wheel's rotation is returned. 
\end{description}

This function is called each time a wheel is drawn. This is done by using \lstinline|translateRotateAndDraw()| as can be seen in \autoref{lst:rotatewheels}

\begin{lstlisting}[language=java,firstnumber=1,caption={Rotating and drawing the wheels.},label=lst:rotatewheels]
	// Rotate and draw the wheels
super.translateRotateAndDraw(this.calculateRotation(this.mediumWheelIndex), this.mediumWheel);
super.translateRotateAndDraw(this.calculateRotation(this.largeWheelIndex), this.largeWheel);
super.translateRotateAndDraw(this.calculateRotation(this.smallWheelIndex), this.smallWheel);
\end{lstlisting}

\begin{description}
\item[Lines 2-4] Each time a wheel has to be drawn, \lstinline|calculateRotation()| with the specific wheelIndex as parameter, is called and the rotation is calculated. The wheel is rotated and then drawn. 
\end{description}

\subsection{Train smoke}

Another element in making it look like the train is moving is the train smoke. While the train is standing still at the station the smoke goes vertically up in the air, but when the train moves the smoke moves horizontally. 

The smoke clouds are reset and updated based on time, as can be seen in \autoref{lst:cloudstime}

\begin{lstlisting}[language=java,firstnumber=1,caption={Smoke clouds getting reset based on time intervals.},label=lst:cloudstime]
//Reset one smoke cloud at the given interval
this.timeSinceLastReset += super.gameData.timeDifference;
if (this.timeSinceLastReset >= this.timeBetweenSmokeClouds) {
    this.timeSinceLastReset = 0f;
    
    if (!super.gameData.isPaused) {
        this.resetOneSmokeCloud();
    }
}

if (!super.gameData.isPaused) {
    //Update position and alpha channels
    this.updateSmokeClouds();
}
\end{lstlisting}

\begin{description}
\item[Line 2] The \lstinline|timeSinceLastReset| is the time since a smoke cloud was last reset and \lstinline|gameData.timeDifference| is the time that has elapsed, in milliseconds. 
\item[Lines 2-9] If the time since last reset is higher or equal to the time between the smoke clouds (this is a variable we have pre-determined), then \lstinline|timeSinceLastReset| is set to 0, and if the game is not paused, then the smoke cloud is reset back to its starting position. 
\item[Lines 11-14] If the game is not paused, then the smoke cloud's coordinates are updated, in order to move the smoke cloud. 
\end{description}

\autoref{lst:cloudstime} showed two methods, \lstinline|resetOneSmokeCloud()| and \lstinline|updateSmokeClouds()|. These are the two methods in charge of updating and resetting the smoke cloud's positions, making it look like they move. They are explained in \autoref{lst:resetonesmokecloud} and \autoref{lst:updatesmokeclouds}

\begin{lstlisting}[language=java,firstnumber=1,caption={Function handling the resetting of a smoke cloud.},label=lst:resetonesmokecloud]
private final void resetOneSmokeCloud() {
    //Increment the reset index
    this.resetIndex = ++this.resetIndex % this.numberOfSmokeClouds;
    
    //Put cloud back to exhaust
    this.coordinates[this.resetIndex].setCoordinate(this.startCoordinate.getX(), this.startCoordinate.getY());
    this.colors[this.resetIndex].setColor(1f, 1f, 1f, 1f);
}
\end{lstlisting}

\begin{description}
\item[Line 3] This resets the index, helping to keep track of which cloud that has to be reset. 
\item[Lines 6-7] This resets the specific cloud back to its starting position and it also resets the color back to its original. 
\end{description}
 
\begin{lstlisting}[language=java,firstnumber=1,caption={Function handling the updating of a smoke cloud.},label=lst:updatesmokeclouds]
private final void updateSmokeClouds() {
    //Updates position and alpha channels
    for (int i = 0; i < this.numberOfSmokeClouds; i++) {
        //Always move smoke vertically, move smoke horizontally relative to the train speed.
        this.coordinates[i].moveX(super.gameData.getPixelMovement());
        this.coordinates[i].moveY(this.ySpeed * super.gameData.timeDifference);
        
        //Fade the smoke
        this.colors[i].alpha -= (1f / (this.timeBetweenSmokeClouds * this.numberOfSmokeClouds)) * super.gameData.timeDifference;
    }
}
\end{lstlisting}

\begin{description}
\item[Lines 3-6] This is in charge of moving the clouds. The clouds are moved on both the x and y axis. The new x coordinate depends on \lstinline|gameData.getPixelMovement()| which is explained in \secref{sec:gamerendering}. The new y coordinate depends on a constant called \lstinline|ySpeed|, and \lstinline|gameData.timeDifference|, which is the time that has elapsed.
\item[Line 9] This fades the smoke clouds. The cloud's original alpha channel is one, and each time the smoke cloud is updated and moved, the alpha color lowered, making the cloud more see-through, to make it look like they disappear up in the air. 
\end{description}

\section{Station}

We have made three different station textures and to ensure an element of surprise in our game, each time we start the game we shuffle the list of stations so the order always will be random, however this only happens when the game is started, so the order within the same game is the same. \autoref{lst:stations} shows this implementation.
 
\begin{lstlisting}[language=java,firstnumber=1,caption={Placing the station correctly.},label=lst:stations]
Collections.shuffle(stations);
LinkedList<StationContainer> stationsQueue = this.getQueue(stations);

float xPosition = -364f; // first platform position

for (int i = 0; i < super.gameData.numberOfStations; i++) {
    StationContainer nextStation = stationsQueue.pop();
    stationsQueue.add(nextStation);
    this.stationPlatformMatrix.addRenderableMatrixItem(nextStation.station, new Coordinate(xPosition + nextStation.xOffset, nextStation.yOffset, 0f));
    
    xPosition += GameData.DISTANCE_BETWEEN_STATIONS;        
    
}
\end{lstlisting}

\begin{description}
\item[Lines 1-2]  The stations are shuffled and added to the queue.
\item[Line 4] \lstinline|xPosition| is the position in which the platform will be drawn. 
\item[Line 7-8] The next station is found and popped off the queue, and added to the end of the queue again to ensure there are always are stations in the queue. 
\item[Line 9] The station is added to the \lstinline|stationPlatformMatrix| and drawn at the correct location. All stations have the same offset to ensure that they are all placed at the same location each time the train stops. 
\item[Line 11] \lstinline|xPosition| is updated with the pre-determined distance between stations to ensure each station has the same distance between them. This is especially important when we have to calculate when the train has to stop. 
\end{description}

\subsection{Stopping position}

In order for us to stop the train at the correct location each time, we have to calculate the stopping position, this is done as shown in \autoref{lst:nextstopping}.

\begin{lstlisting}[language=java,firstnumber=1,caption={Calculating the next stopping position.},label=lst:nextstopping]
public final void calculateStoppingPositions() {        
    //Make new array
    super.gameData.nextStoppingPosition = new float[super.gameData.numberOfStations + 1];
    
    //Calculate all stopping positions
    super.gameData.nextStoppingPosition[0] = GameData.DISTANCE_BETWEEN_STATIONS;
    for (int i = 1; i < super.gameData.numberOfStations; i++) {
        super.gameData.nextStoppingPosition[i] += super.gameData.nextStoppingPosition[i-1] + GameData.DISTANCE_BETWEEN_STATIONS;
    }
}
\end{lstlisting}

\begin{description}
\item[Line 2] An array for the stopping positions is initialized. The array is +1 to ensure an ArrayIndexOutofBoundsException will not occur when calculating the stopping position for the train depot mentioned in \secref{sec:gameidea}.
\item[Line 6] The first stopping position is equal to the pre-determined distance between station. 
\item[Lines 7-9] We loop through the array and calculate the next stopping position for all stations. This is done by adding the distance to the next station to the stopping position of the previous station.
\end{description}

\section{Game Background}

The background for our game is randomly generated for each game session as mentioned in \secref{sec:designgameinterface}

\subsection{Hills}

We have created four different sequences of hills, each sequence consists of four hills in a specific layout. This four sequences are randomly chosen throughout the game and is random each time. It is important to note that every hill texture is only loaded once, even though it is used multiple times in the game. 

HillItem is a class used for making hill sequences. Its constructor takes three parameters, an x-coordinate, a y-coordinate and a texture. The x- and y-coordinates are used for placement of the texture. \autoref{lst:hills} shows one of the hill sequences used in our game.

\begin{lstlisting}[language=java,firstnumber=1,caption={Hill sequence.},label=lst:hills]
HillItem[] hillSequence1 = {
    new HillItem(0f, this.hill_large.getHeight(), this.hill_large),
    new HillItem(this.hill_large.getWidth()+11f, this.hill_medium.getHeight(), this.hill_medium),
    new HillItem(this.hill_large.getWidth()/100f*60f, this.hill_small.getHeight(), this.hill_small),
    new HillItem(this.hill_large.getWidth()+11f + this.hill_medium.getWidth()/100f*65f, this.hill_larger.getHeight(), this.hill_larger)
};
\end{lstlisting}

\begin{description}
\item[Line 2] As a starting point our hills are placed in the bottom left corner, so the first hill has the x-coordinate of 0, and because we draw the texture from the top left corner, we ensure that the hill is placed correctly by making the y-coordinate equal the height of the hill. 
\item[Line 3] The next hill is placed relative to the first hill, the x-coordinate is the width of the hill that came before it plus a constant. The y-coordinate equals the height of the hill again.
\item[Line 4] The x-coordinate for the third hill is a percentage of the width of the first hill. 
\item[Line 5] The last hill is placed relative to the first two hills.
\end{description}

An important thing to note about the placement of the hills is the order in which they are added to the sequence, The hill furthest to the back has to be drawn first, then the second furthest and so on. This is also why the third hill in the sequence is drawn with a smaller x-coordinate than the second hill in the sequence.

\subsection{Trees \& Cows}

The trees and cows are randomly generated as well, however they are implemented used more layers of randomization. \autoref{lst:background} shows the important elements of this.

\begin{lstlisting}[language=java,firstnumber=1,caption={Selection of background.},label=lst:background]
private Texture[] decorationArray = {cow, tree, tree};

drawDecoration = super.gameDrawer.getRandomNumber(0, 2); // Randomly chosen which decoration to draw (cow = 0, tree = 1, tree = 2)
chooseHillSequence = super.gameDrawer.getRandomNumber(0, hillSequences.length-1); // randomly choose which hill sequence to draw
chooseHillToDrawOn = super.gameDrawer.getRandomNumber(1, 4); // randomly choose which hill to draw on (small_hill = 1, medium_hill = 2, large_hill = 3, larger_hill = 4)
    		
for (HillItem hillItem : hillSequences[chooseHillSequence]){
	this.sequence.addRenderableMatrixItem(hillItem.hill, new Coordinate(i + hillItem.x, hillItem.y, 0f));
\end{lstlisting}

\begin{description}
\item[Line 1] We have an array consisting of the cow texture and the tree texture. We chose to have one cow texture and two tree textures so that trees would have a chance of occurring more often. We have several random factors that play in before we find where to place it. 
\item[Line 3] First we randomly choose which of the four hill sequences we want to draw, this is because we do not want cows and trees on every hill as they are meant to be an element of surprise for the player. 
\item[Line 4] Now we know which hill sequence we draw, we have to find out what to draw, this is also chosen randomly. We randomly find a number between 0 and 2 and use this as index in our \lstinline|decorationArray|.
\item[Line 5] We now know what to draw and on which hill sequence to draw it, but we still need to find out which hill in that sequence to draw it on, because we do not want to draw on every hill in the specific sequence. This is again chosen with a random number, which is used to choose a random case. 
\item[Lines 7-8] We add each hill in the sequence to the matrix so they can be drawn in a loop. 
\end{description}

Now it gets a little complicated, so far we have a random number which determines which hill sequence we have to draw, we also have a random number which determines whether a cow or trees should be drawn on this hill sequence and we have a random number deciding which specific hill in this sequence to draw on. 

We have set up four cases, one for each type of hill, \lstinline|hillItem.hill| has to correspond to the case chosen by the random number, if this is the case, then there will be drawn on the hill, if not then nothing is drawn and the loop continues with the next iteration.  

For example if the three numbers that has been chosen are one, one and three. This means that the hill sequence we will draw is hill sequence one and we have to draw a cow on a large hill in this sequence. 

The first hill in this sequence is added to the matrix, and then two scenarios can occur, either \lstinline|hillItem.hill| is a large hill and the cow is drawn on the large hill, or \lstinline|hillItem.hill| is not a large hill and nothing is drawn on the hill and new random numbers are chosen and everything is checked again.\todo{Læs og se om I forstår forklaringen}

\subsection{Clouds \& sun}

The clouds are also implemented with a degree of randomness. There will always be clouds on the sky, however the speed and placement on the sky is random to a certain degree. 

The clouds move on the x-axis with a random velocity between $0.07$ and $0.18$ pixels per millisecond, the clouds are also randomly placed on the top 1/4th of the screen. 

The sun is a stationary texture in the top right corner of the screen. 