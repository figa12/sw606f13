\chapter{OpenGL}
\todo{This Chapter describes OpenGL and which of its functionalities we have used to create our game.}

\section{Analysis}
OpenGL is a 2D and 3D graphics API. It is cross-platform API that specifies a standard for 3D graphics processing hardware. The newest version of OpenGL is $4.3$ which was released August 2012. Android have support for OpenGL for Embedded Systems (OpenGL ES) up to version $2.0$. OpenGL ES $1.1$ is based on the OpenGL $1.5$ specification and is fully backwards compatible with $1.0$, and OpenGL ES $2.0$ is based on OpenGL $2.0$. Android $1.0$ and later versions have support for OpenGL ES $1.0$ and $1.1$ specifications. Android $2.2$ added support for OpenGL ES $2.0$. \citep{androidopengl, khronosopengl, khronosopengles}

\section{Design}
Firstly \ref{fig:terminology} shows how we represent abstract class names and how we denote public or private attributes and methods. A minus denotes a private attribute/method, and a plus denotes a public attribute/method. Italic class names are abstract classes. \todo{Skal det stå her? Det må være C\# UML standard det her.}
\begin{figure}[H]
\centering
\includegraphics[width=0.4\linewidth]{img/terminology.png}%0.1 margin
\caption{The way we make UML described in a figure.}
\label{fig:terminology}
\end{figure}

\subsection*{Choice of OpenGL ES Version}
When using OpenGL ES on the Android platform, the default OpenGL ES version is $1.x$. If you want to use version $2.0$ you need to explicitly write that you are using it. Version $2.0$ should have better performance and more possibilities than the older version, but the Android developers guide says:
\begin{quote}
\textit{"Developers who are new to OpenGL may find coding for OpenGL ES $1.0$/$1.1$ faster and more convenient."} \citep{androidopengl}
\end{quote}
Since we are newbies to OpenGL we have chosen version $1.x$. One big difference between $1.x$ and $2.0$ is that you have to write your own shaders in versions $2.0$, this allows for easier effects customisation. The Train game does not really require much special effects other than simple drawing of textures.

\begin{figure}[H]
\centering
\includegraphics[width=0.9\linewidth]{img/renderables.png}%0.1 margin
\caption{Class diagram of the objects that can be rendered.}
\label{fig:renderables}
\end{figure}


\begin{figure}[H]
\centering
\includegraphics[width=0.9\linewidth]{img/game.png}%0.1 margin
\caption{Class diagram of the objects involved with drawing the game.}
\label{fig:game}
\end{figure}

\section{Implementation}

\todo{Describes the implementation of OpenGL}