Unfortunately due to time limitations we have not been able to implement all the features that our contact person suggested. This Chapter describes some of the improvements that could be made on our application.

\begin{description}
\item[The sun] The sun should have visible sunbeams, making it shine. 

\item[Overwrite games] At the moment it is not possible to overwrite an old game, each time you click the "Gem nyt Spil" button, you save a new instance of a game even if you have changed an old game. There should be a "Gem Spil" button, which allows you to overwrite existing games in order to make it easier for the user to edit and update them.

\item[Sounds on pictogram] The pictograms should have sounds, this could either be when you move them around or when you press them. 

\item[Sound effects in game] More sound effects in the game. This could be the cow mooing when the train passes it, wind blowing or the sound of the train accelerating when it starts from the station.

\item[Customizable] All colors in the game are pre-determined right now. A settings menu where it is possible to change the color of items in the game, for example the color of the train or station. In case more sounds are implemented it could be possible to mute and unmute the different sounds, in order to make it more customizable for the guardian.

\item[Child as train driver] Every child has an avatar picture, it could be funny to have the child's avatar as the train driver. 

\item[Make it clear how to delete a game] It is possible to delete saved games by performing a long click, however this is not mentioned any where. Perhaps a little information button somewhere that explains this could be a good idea. 

\item[Greater variation of background features] The selection of background features is very limited, it can only be cows or trees. A greater selection of items could be implemented, perhaps to include planes, different kinds of animals or even humans. It should still be kept simple, but it would create a greater variation in the background. 

\item[Complete a station in one go] At the moment when creating new stations in the menu section, the user has to click the blank category box, open the \ac{cat}, choose one category pictogram and click the "Send" button, then click the green plus and choose which pictograms to associate with the station.

An easier way for this could be to allow the user to click the blank category box and open PictoSearch and then allow them to choose up to seven pictograms, where the first pictogram would count as the category and the rest would be the pictograms associated with this specific station.

\item[Changing weather] One of the suggestions made by our contact person was that the weather could change while the train moved from one station to another, for example the sun could shine from the first to the second station and it could rain or snow from the second to the third station. 

\item[Layouts disappear when drag starts] At the beginning of each game, and each time you arrive at a station, the first pictogram you drag will make the layouts on the station disappear while the pictogram is being dragged. We have a workaround for this but it creates a lot of overhead. Each time we start a drag event, we draw all the layouts again.

\item[Reduce drag events] While dragging a pictogram the pictogram shakes. Too many dragging events happens while the pictogram is being dragged, this should be reduced. It also uses resources that \ac{opengles} could use for higher frame rate.

\item[Customization optimisation] The lists for children, saved games, and stations were created under pressure of time. When you load a game, a lot of garbage is created from the game that was loaded before that.

\item[Dragging of pictograms in \ac{opengles}] Instead of having regular views and \linebreak \ac{opengles} at the same time, it would be better if everything was implemented in \ac{opengles}. There is some noticeable delay with the communication between the two, especially when pictograms are taken from the views and rendered in \ac{opengles}.

\item[Database] The database was not fully implemented and because of that we save the game configurations to a file. When the database is implemented the following queries will be able to read and write data respectively.
\begin{lstlisting}[language=json,firstnumber=1,caption={JSON query to read application data.},label=lst:jsonread]
{
    "auth": {
        "session": SESSION_STRING
    },
    "action": "read",
    "data": {
        "type": "application",
        "view": "list",
        "ids": null
    }
}
\end{lstlisting}

\begin{lstlisting}[language=json,firstnumber=1,caption={JSON query to write application data.},label=lst:jsonwrite]
{
    "auth": {
        "session": SESSION_STRING
    },
    "action": "link",
    "data": {
        "profile'': PROFILE_ID,
        "link": [
            {
                "type": "application",
                "id": APPLICATION_ID,
                "settings": SETTINGS_STRING
            }
        ]
    }
}
\end{lstlisting}

\item[Usability test] Performing a usability test on our application would help spotting possible flaws in the design and help making it even easier and better for the guardians to use. 

\end{description}
