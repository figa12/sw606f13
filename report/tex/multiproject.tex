Being a part of a multi-project was a new thing for us. The biggest difference was that we were depending on other groups, which was different than what we are used to because we have always been depended on ourselves. Due to the nature of our project, none of the other groups were dependant on us so the decisions we made did not effect anyone else but us, however we depended on other groups and they depended on other groups as well, so a form of controlled decision making process had to be established. 

\begin{description}
\item[Agile developing] Throughout this project we have used an agile developing method, which worked very well for us. Always having a sprint deadline really helped us focus and be on time. 

As a part of the agile developing method we had weekly meetings. They were really helpful to keep track of what all the other groups were doing. It was also great for discussing different aspects of the project. It really helped getting the project organised in the beginning with creation of committees for different assignments, however the further we got with the project, the more obsolete did the meetings seem because all the basic stuff was already determined so it ended up always being the same that was said on these meetings. We felt they started to feel more like a chore than a help. 

With that being said, the meetings have to be there, they are a part of the agile developing method and they are extremely helpful in the start-up phase of the project, however at the end of the project they might only have to be biweekly. 

\item[Committees] The committees were a very good idea, the subjects that concerned all groups were discussed with a representative from each other and then presented at the next weekly meeting where everyone had to vote to accept or decline the idea. A list of the important committees can be found in \secref{sub:committees}

\item[E-mails] During the project we received a lot of different e-mails, and it was sometimes confusing to see if they had anything to do with the project. Late in the project it was suggested that every e-mail that concerned the project would have "GIRAF" in the subject, allowing us to filter the incoming e-mails. This can be highly recommended. 

\item[Roles] In the beginning of the project several different roles were assigned to different people by volunteering. These roles came with an area of responsibility. They worked very well, but unfortunately some groups ended up having more than one role, which meant they had a lot of work which was focused on the multi-project and less on their own. It could be an idea to simply assign these roles to each of the projects before making the groups so that the workload was equally assigned. 
\end{description}

