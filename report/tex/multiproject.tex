Being a part of a multi-project was a new thing for us. The biggest difference was that we were depending on other groups, this has been a significant change compared to other projects. Due to the nature of our project, none of the other groups were dependant on us so the decisions we made did not effect anyone else but us. Some of the other groups were however depended on each other, so a form of controlled decision making process had to be established. 

\begin{description}
\item[Agile development] Throughout the project we have used an agile development method, which worked well for us. Always having a sprint deadline helped us focus and be on time. 

As a part of the agile development method we had weekly meetings. They helped keeping track of what all the other groups were doing. The weekly meetings were also great for discussing different aspects of the project. At the start of the project, the meetings helped getting the project organised with the creation of committees for different assignments. However the further we got with the project, the more unnecessary the meetings became, since all the basic stuff was already determined. This resulted typically with same outcome as the previous meetings. We felt the meetings started to feel more like a chore than like a help. 

With that being said, the meetings have to be there, they are a part of the agile development method and they are extremely helpful in the start-up phase of the project, however at the end of the project they might only have to be biweekly. 

\item[Committees] The committees were a good idea, the subjects that concerned all groups were discussed with a representative from each group, and then presented at the next weekly meeting where everyone had to vote, to either accept or decline the idea. A list of the important committees can be found in \secref{sub:committees}

\item[E-mails] During the project we received a lot of different e-mails, and it was sometimes confusing to see if they had anything to do with the project. Later in the project it was suggested that every e-mail that concerned the project would have to be tagged with "GIRAF" in the subject, allowing us to filter the incoming e-mails. This can be highly recommended. 

\item[Roles] In the beginning of the project several different roles were assigned to different people by volunteering. These roles came with an area of responsibility. They worked very well, but unfortunately some groups ended up having more than one role, which meant they had a lot of work which was focused on the multi-project and less on their own. It could be an idea to simply assign these roles to each of the projects before making the groups so that the workload was equally assigned. 
\end{description}

Some advices for next year; Split all the roles up between all project groups so that the workload is equally assigned. Keep the meetings short and precise, too much unnecessary talk makes the meetings long and frustrating. 