\section{Project introduction}
We implement a game for children with autism on the Android platform, as a part of the GIRAF project.

The inspiration for our game comes from an exercise one of the pedagogues practises with the children. The purpose of the game is to create a dialogue between the child and the pedagogue. The child has to drag pictograms from a train station onto the train wagons and make the train drive. When the train arrives at the next station, the child has to drag the correct pictograms from the train and onto the station. The correct pictograms are decided by the station category.

The category for each station is chosen by the pedagogues by clicking the category picture frame and browse the pictogram database and choose the picture they want to use. After selecting a category they select which pictures they want associated with this station, these are the pictograms the children have drag onto this specific station.

\section{Problem Statement}

Based on our focus in the multiproject, we have come up with the following problem statement:

In what ways can we aid the pedagogues in their work, with children with autism, by digitalizing a physical exercise onto an Android tablet?